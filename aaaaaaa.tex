%% article beginning
\documentclass[a4paper,7pt,fleqn]{article}   %% [A4纸,基本字号10pt,左对齐]{文章类型}

%% 添加宏包
\usepackage{ctex}
\usepackage{authblk} %% 添加论文作者
\usepackage[english]{babel} %% 英语自动断字
\tolerance=1000    \hyphenpenalty=1000   %% 英文断行容忍程度(越高越不断行)
\usepackage{amsmath} %% 普通数学符号
\usepackage{amssymb} %% 加载数学环境
\usepackage[utf8]{inputenc} %% 支持 UTF8 编码
\usepackage{geometry} %% 页面设置
\geometry{a4paper,left=12.5mm,right=12.5mm, top=18mm, bottom=18mm} %% 页面设置 需要 geometry
\usepackage{graphicx} %% 可以插入图片 \includegraphics
\usepackage{sectsty} %% 设置不同层次章节标题的字体
\allsectionsfont{\rmfamily} %% 设置所有章节标题字体 需要 sectsty
\usepackage{float} %% 浮动命令(例如强制图形位置[H])
\usepackage{setspace} %% 允许改变行间距
\linespread{0.97} %% 全文行距0.97*1.2=1.164
\usepackage{tabularx} %% 扩展的表格环境
\usepackage{cite} %% 引用
\usepackage{hyperref} %% 支持超链接
\hypersetup{
	bookmarksopen=true, %% 展开书签
	bookmarksnumbered=true, %% 书签带章节编号
	colorlinks=true, %% 超链接文字带颜色
	linkcolor=red, %% 
	filecolor=blue, %% 引用图片颜色
	urlcolor=cyan, %% 网址链接颜色
	citecolor=cyan, %% 参考文献颜色
}
\usepackage{caption} %% 图片、表格说明
\captionsetup{	figurename=\textbf{Fig.},	tablename=\textbf{Table}}
\usepackage{subcaption} %% 支持子图,并排图
\usepackage{multirow} %% 表格中合并多行
\usepackage{multicol} %% 表格中合并多列
\usepackage{pdfpages} %% 支持引用 pdf 文件
%\usepackage{flushend} %% 双栏文档底部齐平
\usepackage{siunitx} %% 国际制单位
\usepackage{bm} %% 数学公式加粗
\usepackage{fancyhdr} %% 页眉页脚
\pagestyle{fancy}
\fancyhead[L]{\itshape \small N. Ichihara and M. Ueda}
\fancyhead[R]{\itshape \small Composites Part B 253 (2023) 110572}
\fancyfoot[C]{\thepage}
\renewcommand{\headrulewidth}{0pt} %% 分隔线宽度
\renewcommand{\footrulewidth}{0pt}
\usepackage{indentfirst}


\usepackage{titlesec} %% 标题样式修改
\titleformat{\section}{\fontsize{10pt}{13pt} \bfseries \raggedright}{\thesection .}{1ex}{}
\titleformat{\subsection}{\fontsize{9pt}{13pt} \raggedright \itshape}{\thesubsection .}{1ex}{}
\renewcommand{\dblfloatpagefraction}{.9} %% 可让图片占据的位置变大


%% 文档开始
\begin{document}
	\date{}
	
	%% 标题
	\title{\vspace{70pt} \Large \raggedright \textbf{3D-printed high-toughness composite structures by anisotropic topology optimization}}
	
	%% 作者
	\author{\vspace{-5pt}{\bfseries \leftline{Naruki Ichihara\thanks{Corresponding author.\\
					\textit{E-mail address:} \textcolor{cyan}{ ichihara.naruki@nihon-u.ac.jp} (N. Ichihara).},
				Masahito Ueda}}}
	\maketitle
	
	\vspace{-220pt}\noindent
	\begin{minipage}[t]{\textwidth}
		\begin{minipage}[t]{0.85\textwidth}
			\vspace{-\baselineskip} % 水平对齐
			\noindent \rule{\textwidth}{0.05em}
			\vspace{8pt}
			\begin{minipage}[t]{0.2\textwidth}
				\vspace{-\baselineskip} % 水平对齐
				\vspace{6pt}
				\includegraphics[scale=1]{elsevier-logo.jpg}
			\end{minipage}
			\hfill
			\begin{minipage}[t]{0.78\textwidth}
				\vspace{-\baselineskip} % 水平对齐
				\vspace{6pt}
				\begin{center}
					{Contents lists available at \href{www.sciencedirect.com/journal/composites-part-b-engineering}{ScienceDirect}} \par
					\vspace{15pt}				
					{\LARGE \bfseries Composites Part B} \par
					\vspace{15pt}				
					journal homepage: \href{www.elsevier.com/locate/compositesb}{www.elsevier.com/locate/compositesb}
				\end{center}
			\end{minipage}
		\end{minipage}
		\hfill
		\begin{minipage}[t]{0.14\textwidth}
			\vspace{-\baselineskip} % 水平对齐
			\rightline{\includegraphics[scale=1.05]{composites-logo.jpg}~~}
		\end{minipage}
		\vspace{10pt}
		\noindent \rule{\textwidth}{0.36em}
	\end{minipage}
	
	%% 地址
	\vspace{90pt}
	{\small \selectfont \leftline{\rmfamily \emph{\textsuperscript{}{Nihon University, 1-8-14 Kanda-Surugadai, Chiyoda, Tokyo, 101-8308, Japan}}}}
	
	\thispagestyle{empty}
	
	%% 摘要和关键字
	\noindent
	\begin{minipage}[t]{\textwidth}
		\noindent \rule{\textwidth}{0.05em}
		\begin{minipage}[t]{0.3\textwidth}
			\vspace{2pt}
			{A R T I C L E I N F O} \\
			\rule{\textwidth}{0.05em}
			{\fontsize{7pt}{0.8} \selectfont \textit{Keywords:} \\Polymer-matrix composites (PMCs) \\ Carbon fibers \\ Anisotropy \\ 3D printing}
		\end{minipage}
		\hfill
		\begin{minipage}[t]{0.65\textwidth}
			\vspace{2pt}
			{A B S T R A C T} \\
			\rule{\textwidth}{0.05em}
			{\fontsize{7pt}{0.8} \selectfont The toughness of structures is essential to prevent catastrophic failure. This study introduced a design framework to improve the toughness of 3D-printed carbon fiber-reinforced composite structures by local latticing utilizing the intermediate material fraction obtained in the topology optimization. The framework was based on anisotropic topology optimization considering material fraction and material orientation. The optimized results were de-homogenized by the phase field-based technique to determine the 3D printing path. Experimental validations were carried out on a three-point bending beam problem. As a result, it was shown that the framework endowed toughness for the 3D-printed carbon fiber-reinforced composite structure.}\\
		\end{minipage}
		\noindent \rule{\textwidth}{0.05em}
	\end{minipage}
	\vspace{10pt}
	
	
	\begin{multicols}{2} %% 双栏
		
		%% 第一章
		\section{Introduction}
		\label{Introduction}
		A fail-safe design ensures structural safety under unexpected
		destruction. The fail-safe designs have been validated for aerospace
		structures. However, fail-safe capability frequently requires several
		components to prevent catastrophic failures of entire structures, which
		increases the structural weight. A fail-safe monolithic structure offering
		protection from catastrophic failures, instead of structural redundancy,
		can reduce the structural weight. The fail-safe monolithic structure enables
		advanced designs of lightweight primary structures, which further
		expands the use of a fail-safe design for small parts in automotive, robotics,
		and medical applications, such as prostheses.\par    %%
		Monolithic fail-safety can be realized with high-toughness materials.
		However, high stiffness and strength materials such as carbon fiberreinforced
		polymer composites exhibit brittle behavior. Fiber-hybrid
		technique has been studied to improve the toughness \cite{r01}. Glass/carbon
		fiber hybrids \cite{r02} and carbon/carbon fiber hybrids, including
		those with high-modulus/high-strength \cite{r03} and
		low-elongation/high-elongation \cite{r04}, have shown high toughness as
		pseudo-ductility. The toughness improvement of these materials is based
		on a transition between two failure stages. High-modulus fibers bear the
		initial load, and subsequently, high-strength fibers withstand the
		high-level load induced after the failure of the high-modulus fibers.
		However, the pseudo-ductility is restricted by the toughness of the
		constituent materials.\par     %%
		Metastructures are generally composed of complex-lattice internal
		structures that break the mutually exclusive relationship between stiffness
		and toughness \cite{r05,r06}. Loading causes local buckling of the struts in
		their lattices, which absorbs the external work. Successive local buckling
		endows metastructures with high toughness. Many studies have examined
		topology optimization of metastructures \cite{r07,r08,r09,r10,r11,r12,r13}. These optimization
		frameworks divide the process into two parts with different scales;
		first, on a small scale, the mechanical behavior of a metastructure is
		obtained before the optimization process. Subsequently, on the entire
		structure scale, multiple-field optimization determines the optimal
		design parameters. These optimization schemes are focused on isotropic
		materials including metals or polymers because of the geometrical
		complexity of metastructure. However, anisotropic materials such as
		fiber-reinforced composites are beneficial to endow metastructures with
		high mechanical performance-to-weight ratios.\par     %%
		\begin{figure*}[b!]
			\centering
			\includegraphics[scale=1]{fig1.jpg}
			\caption{\small Workflow to generate optimized topology and material path for producing symmetric cross-ply orthotropic lattice geometry. (A) Setting of design domain and boundary conditions and example optimization result. (B) Phase-field of optimization results obtained by solving Eq. \eqref{eq11}. (C) Material path generated from zerolevel contour of phase field. (D) Assembled material path corresponding to 3D printing path.}
			\label{fig1}
		\end{figure*}
		Additive manufacturing (AM) techniques, including stereolithography,
		selective laser sintering, and fused filament fabrication
		(FFF), produce metastructures \cite{r05,r14}. Among these, the FFF method
		prints anisotropic short/continuous fiber-reinforced polymer composites
		by a continuous extrusion process considering material (fiber)
		continuity \cite{r15,r16}. Fiber-reinforced polymer composite lattices are an
		emerging class of metastructures for producing lightweight
		high-performance structures \cite{r17,r18}. The material path plays a significant
		role in the mechanical properties of fiber-reinforced polymer
		composites. For example, a curvilinear material path that conforms to
		the principal loading direction improves its structural stiffness and
		strength \cite{r19,r20,r21}. Thus, material anisotropy as well as material
		continuity must be included in developing metastructures using
		fiber-reinforced polymer composites.\par    %%
		In this study, a homogenization-based topology optimization
		framework to improve the toughness of fiber-reinforced polymer composite
		structures by local latticing was established. The material path, i.
		e., the 3D printing path in the FFF process, was developed based on the
		optimized discrete vector field of the material orientation while maintaining
		the material continuity. In the optimization process, an intermediate
		material fraction was obtained by changing the spacing
		between the paths. The proposed structures achieved high toughness
		after the peak load with remaining the high loading resistance. The
		established framework was applied to a beam structure with a symmetric
		cross-ply orthotropic lattice geometry as an example, and the
		experimental results showed the high-toughness metastructure presented
		fail-safe capability.
		
		
		%% 第二章
		\section{Optimization formulation}
		\label{Optimization formulation}
		The minimum compliance problem was considered in this study to
		obtain high-stiffness structures. The toughness was not included
		explicitly in the optimization process. The toughness was endowed to
		3D-printed carbon fiber-reinforced composite structures by local latticing
		utilizing the intermediate material fraction obtained in the topology
		optimization. The optimal structure was built by optimizing the field
		variables: topology function $\chi (x)$, material fraction related to the unit
		cell geometry $\rho (x) $, and orientation vector $\bm{\theta} (x) $.
		\begin{figure*}[t!]
			\centering
			\includegraphics[scale=1]{fig2.jpg}
			\caption{\small Asymptotic homogenization of symmetric cross-ply orthotropic lattice geometry composite. (A) Material fraction transition due to curvilinear printing path and geometries of unit cells with different material fractions. (B) Definition of unit cell. (C)-(E) Relationships between material fraction and effective stiffness ratios	$C_{11}/C_{11}^{0}$, $C_{12}/C_{12}^{0}$, and $C_{66}/C_{66}^{0}$, respectively.}
			\label{fig2}
		\end{figure*}
		
		\subsection{Topology and material fraction representation}
		The two dependent variable fields for topology design and material
		fraction were used to update the binary external shape and the material
		fraction simultaneously during the optimization process. The topology
		function, $\chi (x)$, was provided in the design domain, \emph{D}, as follows:
		\begin{equation}
			\chi (x) =\left\{\begin{matrix} 0 & \mathrm{for} \forall x\in D\setminus \Omega \\
				1 & \mathrm{for} \forall x\in \Omega\end{matrix}\right.
			\label{eq1}
		\end{equation}
		where $\Omega$ is the material region and $\emph{D}\setminus \Omega$ represents the void region. The
		relaxed topology function, $\tilde{\chi} (x)$, was defined using the implicit design
		parameter for topology, $\varphi (x) \in [0,1] $, and the relaxed Heaviside function,
		$\tilde{H} (\varphi) $, based on the hyperbolic tangent function as follows:
		\begin{equation}
			\tilde{\chi} (x)=\tilde{H}(\varphi) =\frac{\tanh (\beta _{\rho _{\min} } )+\tanh (\beta (\varphi (x)-\rho _{\min} ))}
			{\tanh (\beta _{\rho_{\min}})+\tanh (\beta (1-\rho _{\min})) } 
			\label{eq2}
		\end{equation}
		where $\beta$ controls the smoothness of the projection and $\rho _{\min}$ is the
		threshold of the cutoff value of the implicit design parameter, $\varphi (x)$,
		which is related to the minimum material fraction.\par
		The material fraction, $\rho (x)\in [\rho _{\min},\rho _{\max} ]$, was restricted by lower and
		upper bounds $\rho _{\min}$ and $\rho _{\max}$, respectively. The reduced stiffness tensor 
		$\bm{\mathrm{C}}(x)$ was defined using the above variables and is expressed as follows:
		\begin{equation}
			\mathrm{C}(x)=\mathrm{C}_{\mathrm{void}}+\overset{\sim p}{\chi}(\mathrm{C}(\rho)-\mathrm{C}_{\mathrm{void} })
			\label{eq3}
		\end{equation}
		where $\mathrm{C}_{\mathrm{void}}\approx 0.01\times \mathrm{C}$ represents a small stiffness tensor for the void, p
		is the penalty parameter, which is set as 3, and $\mathrm{C}_{\rho}$ is the effective
		stiffness tensor related to the material fraction. The \emph{p} = 3 is generally
		used to obtain a binary solution for the topology function. The additional
		variable for the material fraction $\rho_{x}$ gives the intermediate
		material fraction. The obtained intermediate material fraction can be
		realized in 3D printing by controlling the spacing between the printing
		paths, which results in local latticing of the optimized structures.\par %%
		In this study, asymptotic homogenization was conducted to obtain
		the relationship between the material fraction, $\rho$, and the effective
		stiffness tensor, $\mathrm{C}(\rho)$, which was investigated experimentally considering
		a symmetric cross-ply orthotropic lattice geometry as an example.
		\begin{figure*}[b!]
			\centering
			\includegraphics[scale=1]{fig3.jpg}
			\caption{\small Optimization results and corresponding 3D-printed structures. (A)-(C) Cases of $\rho_{\min}$ = 0.9, 0.5, and 0.3, respectively.}
			\label{fig3}
		\end{figure*}
		
		\subsection{Material orientation representation}
		A vector representation was chosen for the material orientation field.
		An isoparametric-projection-based method \cite{r22} was used. The orientation
		vector, $\theta (x)\in \mathbb{R}^{2}  $, was provided in the design domain, D. This vector
		should satisfy the following constraint:
		\begin{equation}
			\left \| \theta  \right \| =1\ \mathrm{for} \forall x\in D
			\label{eq4}
		\end{equation}  \par   %%
		However, this position-wise constraint makes solving the optimization
		problem difficult. A relaxed norm constraint was introduced similarly
		to the relaxed topology function, $\tilde{\chi} (x)$.
		\begin{equation}
			\left \| \theta  \right \| \le 1\ \mathrm{for} \forall x\in D
			\label{eq5}
		\end{equation}
		The implicit design parameters for the orientation vector,
		$\vartheta (x)\in [-1,1]^{2}$, were provided. These parameters were converted to the
		orientation tensor using the isoparametric projection, $\mathrm{N}: \mathbb{R}^{2}\to \mathbb{R}^{2}$, as follows:
		\begin{equation}
			\theta(x)=\mathrm{N}(\vartheta(x))
			\label{eq6}
		\end{equation}    \par     %%
		An isoparametric projection N that converts a box constraint to a
		circle constraint was used. For details of the formulation of the isoparametric
		projection, refer to Ref. \cite{r22}. A rotated stiffness tensor was
		defined as follows using the reduced stiffness tensor, C(\emph{x}), in Eq. \eqref{eq3}:
		\begin{equation}
			\mathrm{C}^{\mathrm{rot}}(x)=\mathbf{R}(\angle \theta)\mathrm{C}(x)\mathbf{R}^{\mathrm{T}}(\angle \theta)
			\label{eq7}
		\end{equation}
		where \textbf{R} is the rotation tensor.
		
		\subsection{Optimization problem}
		All design variables were vectorized to a single design vector $d(x)$ as follows:
		\begin{equation}
			d(x)=\begin{bmatrix}\varphi  \\\rho  \\\vartheta \end{bmatrix}
			\label{eq8}
		\end{equation}   \par
		The Helmholtz equation was used to regularize the design vector,
		$d(x)$, with a filter radius R \cite{r23} as follows:
		\begin{equation}
			-R^{2} \bigtriangledown ^{2} \hat{d} +\hat{d} =d
			\label{eq9}
		\end{equation}
		
		where $\hat{d}$ represents the regularized design vector. This filtering method
		also improved the continuity of the material orientation.  \par  %%
		The elastic compliance minimization (i.e., maximizing stiffness)
		problem was defined, using the above equations, as follows:
		\begin{subequations}
			\begin{align}
				\min_{d} F\equiv \int_{D}\varepsilon : \mathrm{C}^{\mathrm{rot}}: \varepsilon\ \mathrm{d}\Omega
				\label{eq10a}\\
				\mathrm{Subject to:}\\
				d\in [0,1]\times [\rho _{\min},\rho _{\max}]\times [-1,1]^{2}   
				\label{eq10b} \\
				G(d)\equiv \int_{D}\tilde{\chi}\rho\ \mathrm{d}\Omega -\bar{M} \le 0
				\label{eq10c}\\
				\left\{\begin{matrix}-\bigtriangledown \cdot \mathbf{\sigma}  =0\ \mathrm{in}\ D 
					\\\mathbf{u} =0\ \mathrm{on}\ \mathit{\Gamma}_{D} 
					\\\mathbf{\sigma}  \cdot \mathbf{\mathrm{n}} =t \ \mathrm{on}\ \mathit{\Gamma}_{N} \end{matrix}\right.
				\label{eq10d}\\
				\sigma =\mathrm{C} ^{\mathrm{rot}} \mathrm{\varepsilon}
				\label{eq10e}\\
				\mathrm{\varepsilon } =\frac{1}{2}(\bigtriangledown u+(\bigtriangledown u)^{\mathrm{T}})
				\label{eq10f}
			\end{align}
		\end{subequations}
		where $\bar{M} $ is the upper boundary of the total material amount, \emph{u} is the
		displacement vector fixed as zero on the Dirichlet boundary, $\mathit{\Gamma _{D}}$, $\bm{t}$ is a traction vector defined on Neumann boundary $\mathit{\Gamma _{N}}$, $\bm{n}$ is the normal vector, $\sigma$ is the stress tensor, and $\varepsilon$ is the strain tensor. By optimization using finite element analysis (FEA), the design variables were obtained as a discretized vector field, and the results are shown in Fig. \ref{fig1}A.
		\begin{figure*}[t!]
			\centering
			\includegraphics[scale=1]{fig4.jpg}
			\caption{\small Results of three-point bending tests. (A) Force-defection curves, (B)-(E) Printing path of $\rho_{\min}$ = 0.9 (0-1 structure), $\rho_{\min}$ = 0.5, $\rho_{\min}$ = 0.3, and $\pm45^{\circ}$ uniform grid, respectively. (F) Enlarged force-defection curves to represent the initial stiffness. (G) Comparisons of the stiffness between FEA and experimental results. (H)Comparison of residual toughness.}
			\label{fig4}
		\end{figure*}
		
		%% 第三章
		\section{Print path generation based on optimization result}
		\label{Print path generation based on optimization result}
		A de-homogenization approach was used to build a composite variable
		lattice structure based on the optimization results because the
		discretized vector representation homogenized the design variables.
		Various de-homogenization techniques have been proposed, including
		the projection approach \cite{r24}, the spinodoid metamaterial approach \cite{r25,r26}
		, the explicit geometry approach \cite{r14}, and deep learning-based approaches
		\cite{r27}. However, these approaches are unsuitable for FFF, owing
		to the geometrical complexity of branching and holes.\par    %%
		In this study, a printing path was directly de-homogenized from the
		optimization results, instead of projecting the complex geometry. The
		printing path was generated using the phase-field approach, in which a
		stripe pattern was derived based on the orientation vectors, which is
		shown in Fig. \ref{fig1}B \cite{r21}. Two stripe patterns, that follow the optimal
		vector field and its orthogonal direction, are needed for the orthotropic
		lattice geometry. This method enabled the generation of a material path
		for a variable lattice with material continuity. The following equation
		was used to obtain the \emph{i}-th layer of the phase field $\varphi_{i}$, for the variable lattices:
		\begin{equation}
			\frac{\partial \varphi _{i} }{\partial t} =-(\bigtriangledown ^{2}+k^{2})^{2}\varphi _{i}+
			2q^{2}\bigtriangledown \cdot \left \{ (\theta _{i}\otimes \theta _{i})\bigtriangledown \varphi _{i} \right \}
			+\varepsilon \varphi _{i}-\varphi_{i}^{3} 
			\label{eq11}
		\end{equation}
		where $\theta_{i}$ denotes the optimized orientation vector field of 
		
		the \emph{i}-th layer. \emph{q} = 2 and $\varepsilon$ = 20 are the parameters. The wavenumber, \emph{k}, is a function of the material fraction, $\rho$, and the width of the material path $w_{0}$, as follows:
		\begin{equation}
			k=f(\rho ;w_{0})
			\label{eq12}
		\end{equation}   \par     %%
		The specific form of function \emph{f} is defined in Section \ref{Experimental}. The material
		path was obtained as a zero-level contour of the phase field, $\phi_{i}$ (Fig. \ref{fig1}C).
		Finally, the material path was cut using the boundary shape, which is
		shown in Fig. \ref{fig1}D. The material path was used as the 3D printing path in the FFF process.
		
		%% 第四章
		\section{Experimental}
		\label{Experimental}
		
		\subsection{Effective stiffness tensor of symmetric cross-ply orthotropic lattice}
		The effective stiffness tensor, $\mathrm{C}(\rho)$, must be related to the material
		fraction, $\rho$, in the optimization process. An asymptotic homogenization
		method was used to obtain the effective stiffness tensor as a function of
		the material fraction. In this study, a symmetric cross-ply orthotropic
		lattice geometry was adopted as an example of a variable lattice, and it is
		shown in Fig. \ref{fig2}A. Orthogonal anisotropy was assumed in the FFF process
		of short carbon fiber-reinforced polymer composites, and the principal
		material direction coincided with the printing direction (Fig. \ref{fig2}B). The
		material fraction, $\rho$, was a variable and represented by the printing path
		width, $w_{0}$, and the length, \emph{L}, of a unit cell.
		\begin{equation}
			\rho(L)=\frac{w_{0}}{L}
			\label{eq13}
		\end{equation}
		where the printing path width, $w_{0}$, was set as a constant and \emph{L} was
		changed to correspond to the material fraction, $\rho$. The wave number, \emph{k},
		in Eq. \eqref{eq12} was defined based on Eq. \eqref{eq13} as follows:
		\begin{equation}
			k=\frac{\pi \rho \chi}{w_{0}} 
			\label{eq14}
		\end{equation}   \par   %%
		The stacking sequence was indicated using the optimal orientation $\theta$ as follows:
		\begin{equation}
			\theta_{i}=[\angle \theta /\angle \theta +90^{\circ}]_{(n/2)s}
			\label{eq15}
		\end{equation}
		where \emph{n} is the total number of stackings and s represents symmetry. \par   %%
		Two-dimensional optimization was considered in this study. A unit
		cell of the symmetric cross-ply orthotropic lattice geometry exhibited
		the same material properties in the two orthogonal directions. Therefore,
		the independent components of the stiffness tensor were three;
		$C_{11}(=C_{22})$, $C_{12}$, and $C_{66}$. Fig. \ref{fig2}C-E presents the relationships between
		the stiffness ratios, $C_{11}/C_{11}^{0}$, $C_{12}/C_{12}^{0}$, and $C_{66}/C_{66}^{0}$, and the material
		fraction, $\rho$, where superscript 0 represents solid stiffness. The short
		carbon fiber-reinforced polyamide 12 was used in this study. The
		longitudinal and transverse modulus were obtained by the tensile tests
		of the 3D-printed specimens and listed in Table \ref{tab1}. The shear modulus
		was obtained by fitting between FEA and the experimental results of the
		$\pm45^{\circ}$ uniform grid structure. Here, the small shear stiffness was used in
		the optimization to promote the orientation of materials to the principal
		loading direction. Tensile tests were conducted on 3D-printed uniform
		grids of short carbon fiber-reinforced polyamide 12 composites with
		several material fractions to validate the calculation results, which are
		shown in Fig. \ref{fig2}C. The experimental results supported the accuracy of the
		calculations.
		
		\subsection{Numerical implementation}
		The optimization problem expressed in Eq. \eqref{eq10a} was resolved using
		the FEA method. The FEA method was used to determine the total strain
		energy by solving the elastic problem using the current design variables.
		Subsequently, a sensitivity-based optimization algorithm (the method of
		moving asymptotes) iteratively updated the design variables. FEniCS, an
		FEA framework in Python, was used for all analyses, including asymptotic
		homogenization. The sensitivities of the design variables were
		calculated using the pyadjoint algorithm.
		\begin{table}[H]
			\caption{Material properties.}
			\small
			\begin{tabular*}{90mm}{p{40mm}p{40mm}}
				\hline
				&Short carbon fiber-reinforced polyamide 12\\
				\hline
				Young's modulus along printing direction & 4848\\
				& \\
				$E_{11}$ [MPa] & \\
				\hline
				Young's modulus along transverse direction & 725\\
				& \\
				$E_{12}$ [MPa] & \\
				\hline
				In-plane shear stiffness & 1367($400^{*}$)\\
				& \\
				$G_{11}$ [MPa] & \\
				\hline
				In-plane Poisson's ratio & 0.33\\
				& \\
				$\nu_{12}[-]$ [MPa] & \\
				\hline
			\end{tabular*}
			(\raggedright $^{*}$ was used in the optimization).
			\label{tab1}
		\end{table}
		
		\subsection{Materials and method}
		A short carbon fiber-reinforced polyamide 12 filament (PolyMide
		PA12-CF, Polymaker) was used for 3D printing. A three-point bending
		beam configuration was chosen for experimental validation. All specimens
		were fabricated using an FFF-based 3D printer (Composer A4,
		Anisoprint) with the nozzle heated to \SI{270 }{\degreeCelsius} and the building plate to
		\SI{60 }{\degreeCelsius}. The printing path width was set as 0.5 mm, and the thickness of
		each layer was 0.2 mm. The total thickness of each specimen was 10 mm (50 layers).
		\begin{figure*}[t!]
			\centering
			\includegraphics[scale=1]{fig5.jpg}
			\caption{\small Image sequences during three-point bending tests. (A) $\rho_{\min}$ = 0.9 (0-1 structure), (B) $\rho_{\min}$ = 0.5, (C) $\rho_{\min}$ = 0.3, and (D) $\pm45^{\circ}$ uniform grid.}
			\label{fig5}
		\end{figure*}
		\begin{figure*}[t!]
			\centering
			\includegraphics[scale=1]{fig6.jpg}
			\caption{\small Material discontinuity defects when $\rho_{\min}$ = 0.3. (A) Deformation visualized by DIC technique. (B) Enlarged view of deformation.}
			\label{fig6}
		\end{figure*}
		
		\subsection{Optimization results}
		Fig. \ref{fig3}A-C shows the optimized results and the 3D-printed beams,
		which were obtained using different lower bounds of the material
		fraction, $\rho_{\min}$ = 0.9, 0.5, and 0.3, under the same constraint on the
		material amount, $\bar{M} $ = 0.4. Stringing was developed due to the material
		oozing out of the nozzle while the print head was traveling to another
		point. This defect did not affect the mechanical performance of the
		structures. The initial material fractions and the orientations were set as
		$\rho_{\min}$ and $45^{\circ}$, respectively; the $0^{\circ}$ direction corresponding to the x1 direction,
		which is shown in Fig. \ref{fig1}A. The optimized results of the structural
		compliance of the three structures are displayed in Fig.\ref{fig3}, where all
		structures show almost the same values because the objectives
		converged in all cases. The printing paths were directed to the optimized
		orientation. The low material fraction was realized by spacing the
		printing paths, which developed locally latticed structures. The high
		$\rho_{\min}$ = 0.9 provided 0-1 structures.    \par    %%
		Latticed structures were obtained by adopting lower bounds of the
		material fraction, $\rho_{\min}$ = 0.5 and 0.3, in the optimization process. The
		high toughness was expected for the latticed structures because of the
		successive small-scale local buckling of the lattice geometry due to
		loading. The improvement of toughness was investigated experimentally,
		as discussed in the following section.
		
		\subsection{Assessment of toughness improvement by three-point bending test}
		Three-point bending tests were conducted on 3D-printed fiberreinforced
		polymer composite variable-lattice structures. In the case of
		$\rho_{\min}$ = 0.5, the loading-unloading tests were performed after the peak
		load to show the recoverability. Fig. \ref{fig4}A shows the load-deflection curves
		of the structures with various lower bounds of the material fraction and
		a $\pm45^{\circ}$ uniform grid structure as a benchmark. For each curve, A 75\%
		peak load, peak load, and 15\% reduction after the peak load points were
		plotted. The deformation of each structure during the bending tests is
		shown in Fig. \ref{fig5}.    \par    %%
		The 0-1 structure showed the highest stiffness and peak load and the
		$\pm45^{\circ}$ uniform grid structure showed the lowest. Fig. \ref{fig4}G compares the
		stiffness between the FEA and experimental results. In the cases of 0-1
		structure and $\pm45^{\circ}$ uniform grid structure, the FEA results predicted the
		experimental result with high accuracy (the error is less than 3.7\%).
		However, the experimental results with $\rho_{\min}$ = 0.5 and 0.3 showed
		smaller stiffness as compared to the FEA results. This error was attributed
		to the detached printing path that was not considered in the optimization
		process. Fig. \ref{fig6}A shows the deformation of the 3D-printed
		structure with $\rho_{\min}$ = 0.3, as visualized by the digital image correlation
		(DIC) technique, and Fig. \ref{fig6}B shows an enlarged view of the deformation.
		In the large deformation regions represented as a red color,
		detached material paths were found, which caused discontinuous load
		transfer. Local large deformations occurred at the points associated with
		a reduction in the stiffness of the entire structure. This print-path defect
		might be caused by the divergence of the optimized material orientation
		fields. Some papers proposed the reduction of the total divergence of the
		material orientation, termed divergence-free vector field methods \cite{r28}.
		However, these constraints may strongly restrict the solution space,
		therefore, the result may fall into local minima.     \par    %%
		The 0-1 structure presented brittle failure after the peak load, with
		abrupt crack propagation (Fig. \ref{fig5}A). In contrast, with $\rho_{\min}$ = 0.3 and 0.5,
		the ductility was observed after the peak loading. Local buckling
		developed because of the latticed structure, and buckling failure progressed
		after the peak load (Fig. \ref{fig5}B and C). The successive local buckling
		behavior led to the brittle failure of the entire structure. Unloading-
		reloading tests were conducted after the peak loading, in which the
		load recovered to almost unloaded points.     \par    %%
		The residual toughness was calculated as an area between peak load
		point to 15\% reduction point after peak load to indicate the monolithic
		fail-safe capability. The residual toughness indicates how much energy
		can be held after peak loading with remaining high loading. Fig. \ref{fig4}H
		shows the residual toughness of each structure. The lattice structures
		showed improved residual toughness as compared to the 0-1 structure.
		The local latticing utilizing the intermediate material fraction obtained
		in the topology optimization improved the residual toughness of the 3Dprinted
		fiber-reinforced polymer composite.
		
		%% 第五章
		\section{Conclusions}
		\label{Conclusions}
		A homogenization-based topology optimization framework was
		established for fiber-reinforced polymer composite variable lattices. A
		3D printing path was generated by considering material continuity
		based on the optimized results of the topology and material orientation.
		The proposed method was used on a beam structure with a symmetric
		cross-ply orthotropic lattice geometry. The intermediate material fraction
		was realized by spacing the printing path during 3D printing.
		Unloading-reloading tests after the peak load also validated the recoverability
		of load resistance. The latticing of the intermediate material
		fraction regions improved the toughness of 3D-printed carbon fiberreinforced
		polymers.
		
		%% 各个作者对本文的贡献
		\section*{CRediT authorship contribution statement}
		\textbf{N. Ichihara:} Conceptualization, Methodology, Investigation,
		Validation, and Writing - Original Draft. \textbf{M. Ueda:} Conceptualization,
		Writing - Review \& Editing, and Supervision.
		
		%% 出版利益说明
		\section*{Declaration of competing interest}
		The authors declare that they have no known competing financial
		interests or personal relationships that could have appeared to influence
		the work reported in this paper.
		
		%% 数据支持
		\section*{Data availability}
		Data will be made available on request.
		
		%% 基金支持
		\section*{Acknowledgments}
		Funding: This research did not receive any specific grants from
		funding agencies in the public, commercial, or not-for-profit sectors.
		
		%% 参考文献
		\begin{thebibliography}{99} \small
			\bibitem[1]{r01}\href{https://doi.org/10.1080/09506608.2018.1467365}{Swolfs Y, Verpoest I, Gorbatikh L. Recent advances in fiber-hybrid composites: materials selection, opportunities and applications. Int Mater Rev 2019;64:181-215.}
			\bibitem[2]{r02}\href{https://doi.org/10.1016/j.compositesa.2015.02.014}{Yu H, Longana ML, Jalalvand M, Wisnom MR, Potter KD. Pseudo-ductility in intermingled carbon/glass hybrid composites with highly aligned discontinuous fibers. Compos Part A Appl Manuf 2015;73:35-44.}
			\bibitem[3]{r03}\href{https://doi.org/10.1016/j.compositesb.2020.108213}{Sapozhnikov SB, Swolfs Y, Lomov SV. Pseudo-ductile unidirectional high modulus/high strength carbon fiber hybrids using conventional ply thickness prepregs. Compos B Eng 2020;198:108213.}
			\bibitem[4]{r04}\href{https://doi.org/10.1016/j.compositesb.2016.11.049}{Czél G, Jalalvand M, Wisnom MR, Czigány T. Design and characterization of highperformance, pseudo-ductile all-carbon/epoxy unidirectional hybrid composites. Compos B Eng 2017;111:348-56.}
			\bibitem[5]{r05}\href{https://doi.org/10.1038/s41563-021-01182-1}{Shaikeea AJD, Cui H, O'Masta M, Zheng XR, Deshpande VS. The toughness of mechanical metamaterials. Nat Mater 2022;21:297-304.}
			\bibitem[6]{r06}\href{https://doi.org/10.1016/j.jmps.2021.104341}{Yin S, Guo W, Wang H, Huang Y, Yang R, Hu Z, et al. Strong and tough bioinspired additive-manufactured dual-phase mechanical metamaterial composites. J Mech Phys Solid 2021;149:104341.}
			\bibitem[7]{r07}\href{https://doi.org/10.1109/TVCG.2017.2655523}{Wu J, Aage N, Westermann R, Sigmund O. Infill optimization for additive manufacturing—approaching bone-like porous structures. IEEE Trans Visual Comput Graph 2018;24:1127-40.}
			\bibitem[8]{r08}\href{https://doi.org/10.1126/sciadv.abf4838}{Sanders ED, Pereira A, Paulino GH. Optimal and continuous multilattice embedding. Sci Adv 2021;7:eabf4838.}
			\bibitem[9]{r09}\href{https://doi.org/10.1007/s00158-021-02874-7}{Groen JP, Thomsen CR, Sigmund O. Multi-scale topology optimization for stiffness and de-homogenization using implicit geometry modeling. Struct Multidiscip Optim 2021;63:2919-34.}
			\bibitem[10]{r10}\href{https://doi.org/10.1016/j.compstruct.2021.114768}{Jung T, Lee J, Nomura T, Dede EM. Inverse design of three-dimensional fiber reinforced composites with spatially-varying fiber size and orientation using multiscale topology optimization. Compos Struct 2022;279:114768.}
			\bibitem[11]{r11}\href{https://doi.org/10.1016/j.cma.2020.113220}{Kim D, Lee J, Nomura T, Dede EM, Yoo J, Min S. Topology optimization of functionally graded anisotropic composite structures using homogenization design method. Comput Methods Appl Mech Eng 2020;369:113220.}
			\bibitem[12]{r12}\href{https://doi.org/10.1002/nme.5575}{Groen JP, Sigmund O. Homogenization-based topology optimization for highresolution manufacturable microstructures. Int J Numer Methods Eng 2018;113: 1148-63.}
			\bibitem[13]{r13}\href{https://doi.org/10.1007/s00158-021-02881-8}{Wu J, Sigmund O, Groen JP. Topology optimization of multi-scale structures: a review. Struct Multidiscip Optim 2021;63:1455-80.}
			\bibitem[14]{r14}\href{https://doi.org/10.1016/j.addma.2021.101920}{Lee J, Kwon C, Yoo J, Min S, Nomura T, Dede EM. Design of spatially-varying orthotropic infill structures using multiscale topology optimization and explicit dehomogenization. Addit Manuf 2021;40:101920.}
			\bibitem[15]{r15}\href{https://doi.org/10.1038/srep23058}{Matsuzaki R, Ueda M, Namiki M, Jeong T-K, Asahara H, Horiguchi K, et al. Three-dimensional printing of continuous-fiber composites by in-nozzle impregnation. Sci Rep 2016;6:23058.}
			\bibitem[16]{r16}\href{https://doi.org/10.1016/j.cjmeam.2022.100016}{Tian X, Todoroki A, Liu T, Wu L, Hou Z, Ueda M, et al. 3D printing of continuous fiber reinforced polymer composites: development, application, and prospective. Chin J Mech Eng Addit Manuf Front 2022;1:100016.}
			\bibitem[17]{r17}\href{https://doi.org/10.1002/adma.201401804}{Compton BG, Lewis JA. 3D-printing of lightweight cellular composites. Adv Mater 2014;26:5930-5.}
			\bibitem[18]{r18}\href{https://doi.org/10.1016/j.compositesb.2022.109894}{Estakhrianhaghighi E, Mirabolghasemi A, Shi J, Lessard L, Akbarzadeh AH. Architected cellular fiber-reinforced composite. Compos B Eng 2022;238:109894.}
			\bibitem[19]{r19}\href{https://doi.org/10.1016/j.compscitech.2019.107905}{Sugiyama K, Matsuzaki R, Malakhov AV, Polilov AN, Ueda M, Todoroki A, et al. 3D printing of optimized composites with variable fiber volume fraction and stiffness using continuous fiber. Compos Sci Technol 2020;186:107905.}
			\bibitem[20]{r20}\href{https://doi.org/10.1016/j.compstruct.2020.112956}{Matsuzaki R, Mitsui K, Hirano Y, Todoroki A, Suzuki Y. Optimization of curvilinear fiber orientation of composite plates and its experimental validation. Compos Struct 2021;255:112956.}
			\bibitem[21]{r21}\href{https://doi.org/10.1016/j.compositesb.2022.109626}{Ichihara N, Ueda M. 3D-print infill generation using the biological phase field of an optimized discrete material orientation vector field. Compos B Eng 2022;232: 109626.}
			\bibitem[22]{r22}\href{https://doi.org/10.1002/nme.4799}{Nomura T, Dede EM, Lee J, Yamasaki S, Matsumori T, Kawamoto A, et al. General topology optimization method with continuous and discrete orientation design using isoparametric projection. Int J Numer Methods 2015;101:571-605. Eng.}
			\bibitem[23]{r23}\href{https://doi.org/10.1002/nme.3072}{Lazarov BS, Sigmund O. Filters in topology optimization based on Helmholtz-type differential equations. Int J Numer Methods Eng 2011;86:765-81.}
			\bibitem[24]{r24}\href{https://doi.org/10.1007/s00158-020-02681-6}{Stutz FC, Groen JP, Sigmund O, Bærentzen JA. Singularity aware dehomogenization for high-resolution topology optimized structures. Struct Multidiscip Optim 2020;62:2279-95.}
			\bibitem[25]{r25}\href{https://doi.org/10.1038/s41524-020-0341-6}{Kumar S, Tan S, Zheng L, Kochmann DM. Inverse-designed spinodoid metamaterials. Npj Comput Mater 2020;6:1-10.}
			\bibitem[26]{r26}\href{https://doi.org/10.1016/j.cma.2021.113894}{Zheng L, Kumar S, Kochmann DM. Data-driven topology optimization of spinodoid metamaterials with seamlessly tunable anisotropy. Comput Methods Appl Mech Eng 2021;383:113894.}
			\bibitem[27]{r27}\href{https://doi.org/10.1016/j.cma.2021.114197}{Elingaard MO, Aage N, Bærentzen JA, Sigmund O. De-homogenization using convolutional neural networks. Comput Methods Appl Mech Eng 2022;388:114197.}
			\bibitem[28]{r28}\href{https://doi.org/10.1016/j.cma.2020.113574}{Tian Y, Pu S, Shi T, Xia Q. A parametric divergence-free vector field method for theoptimization of composite structures with curvilinear fibers. Comput Methods Appl Mech Eng 2021;373:113574.}
			
		\end{thebibliography}
		
	\end{multicols}
	
\end{document}